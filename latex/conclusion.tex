\setlength{\parindent}{15.0pt} % algún comando dejó en cero el parindent
En este trabajo pudimos no solo modelar el problema planteado, sino
apreciar y aprovechar las propiedades del mismo para así resolverlo con los
métodos estudiados observando también las características de ellos.

Así mismo, cabe destacar que al realizar operaciones con aritmética finita,
tanto para la solución de los sistemas como para los valores evaluados en cada
interpolador, no podemos garantizar que los resultados obtenidos sean exactos,
pero dado que realizamos varias instancias de prueba con distintas metodologías,
pudimos ver que los valores que obtuvimos eran coherentes a su contexto.

Todos los métodos de interpolación realizados en este trabajo cumplieron con la
tarea de generar en mayor o menor medida el efecto de cámara lenta que
buscábamos, aunque cada uno con sus características particulares. Mediante la
experimentación pusimos bajo la lupa cada interpolador utilizado para así poder
compararlos entre sí.

Comenzamos nuestros experimentos probando el correcto funcionamiento de los
métodos implementados. Aquí además de ver mediante las cotas de precisión que
todos los métodos cumplían la tarea de interpolar las funciones predeterminadas
pudimos tener una primera visión sobre cómo se comportaba cada interpolador. El
más básico, por vecinos, efectivamente demostró tener la mayor cota de error
distanciándose ampliamente del resto, mientras que la interpolación lineal
obtuvo resultados similares a los obtenidos con splines. En este experimento en
particular, la interpolación mediante splines probó ser la de menor error,
superando a la lineal para funciones cuadráticas y cúbicas, donde a su vez se
pudo observar su relación con la interpolación por splines de a bloques, que con
bloques más grandes su cota se aproximaba a la del spline standard.

En la siguiente prueba, donde lo que se analizó fue el error producido al
eliminar cuadros del video original e intentar recrearlos con nuestros
interpoladores, los resultados fueron similares al experimento anterior. Con
ambas fórmulas de cálculo de error (ECM y PSNR) se pudo ver cómo vecinos era el
método que peor se comportaba mientras que lineal y splines estuvieron
emparejados a lo largo de todas las instancias de prueba realizadas.

Finalmente, la búsqueda de artifacts en el video producido, de anomalías y
calidad final a nivel visual para el usuario resultó ser el punto determinante
para decidir cuál de los sistemas propuestos mejor se desempeña en la tarea de
proveer un efecto de cámara lenta. Los resultados acá dieron un giro completo
con respecto a las conclusiones que veníamos realizando respecto a las métricas
estudiadas. Para empezar, la interpolación de a vecinos dio mejores
resultados que la lineal en términos de cantidad de artifacts y calidad general
del video producido. El método por splines fue el que mejor llegó a producir la
sensación de cámara lenta, con algunas anomalías pero no lo suficientemente
significantes como las vistas en la interpolación lineal. Esto creemos que se
podría explicar con el hecho de que las métricas seleccionadas para decidir qué
interpolador era mejor estaban demasiado atadas a que interpolaran
correctamente sin contemplar cómo resultaba visualmente esta interpolación.

Por último, podemos mencionar como posibles desarrollos a futuro la búsqueda de
métricas que se centren más en el aspecto visual de los interpoladores,
distintas formas de interpolación o directamente métodos completamente
distintos, como lo sería considerar la aplicación de cuadrados mínimos, para
seguir comparando y buscando que se generen los menores artifacts posibles.
