En esta sección, se detallan los diferentes experimentos que realizamos para medir el funcionamiento, la eficiencia y calidad de resultados, tanto de forma cuantitativa como cualitativa, de los metodos implementados.
Para lograr tal fin realizamos los siguientos tipos de experimentos:
\begin{itemize}
  \item Funcionamiento de los metodos implementados.
  \item Determinacion del tamano de bloque del metodo Spline.
  \item Medicion del ECM y PSNR de los metodos.
  \item Medicion de los tiempos de ejecucion de los metodos.
  \item Analisis cualitativos de los metodos, fenomeno de artifacts.
  \item VER DE AGREGAR O SACAR EXPERIMENTOS.
\end{itemize}

\subsection{Funcionamiento de los metodos implementados}
En este experimento nuestro objetivo fue asegurarnos el correcto funcionamiento de nuestra implementacion de la interpolacion fragmentaria lineal y spline. Con ese fin, generamos diversas instancias de distintos tamano y comparamos los resultados obtenidos con los resultados de \textit{OCTAVE} utilizando la funcion \textit{interpol1} \footnote{\url{https://www.gnu.org/software/octave/doc/interpreter/One_002ddimensional-Interpolation.html}} con los parametros acorde.
DETALLE DE LA INSTANCIAS?
A continuacion mostramos un grafico en donde la diferencia entre pixels

\subsection{Determinacion del tamano de bloque del metodo Spline}


\subsection{Determinacion del tamano de bloque del metodo Spline}


\subsection{Medicion del ECM y PSNR de los metodos.}


\subsection{Medicion de los tiempos de ejecucion de los metodos}


\subsection{Analisis cualitativos de los metodos, fenomeno de artifacts.}

\subsubsection{Artifacts: Movimientos bruscos}

\subsubsection{Artifacts: Imagenes fijas}

\subsubsection{Artifacts: Pantalla negra}
