\subsection{Interpolación por vecinos}

Este método consiste en reemplazar los cuadros intermedios a ser rellenados por el cuadro original mas cercano en el tiempo.
Es decir, dados los cuadros del video sin camara lenta, generamos otro video en camara lenta copiando los cuadros originales de la siguiente manera:

% \begin{bytefield}{16}
% \wordbox{1}{A 16-bit field} \\
% \bitbox{8}{8 bits} & \bitbox{8}{8 more bits} \\
% \wordbox{2}{A 32-bit field. Note that text wraps within the box.}
% \end{bytefield}

Sean Frame1 y Frame2 dos cuadros consecutivos del video original:

\begin{centering}
\begin{bytefield}{8}
\bitbox{4}{Frame1} & \bitbox{4}{Frame2}
\end{bytefield}
\end{centering}

Si queremos ahora 6 cuadros entre cada 2 del archivo original lo transformamos a:

\begin{centering}
\begin{bytefield}{32}
\bitbox{4}{Frame1} & \bitbox{4}{Frame1} & \bitbox{4}{Frame1} & \bitbox{4}{Frame1} & \bitbox{4}{Frame2} & \bitbox{4}{Frame2} & \bitbox{4}{Frame2} & \bitbox{4}{Frame2}
\end{bytefield}
\end{centering}

El pseudocódigo sería el siguiente:

\begin{lstlisting}
Sean W,H,I el ancho, alto y la cantidad de frames del video original
Sea video[W][H][I] el triple vector de numeros enteros que representa el video original
Sea K la cantidad de frames que queremos agregar entre cuadro y cuadro
Crear un triple vector de enteros new_video[W][H][I+(I-1)*K]
Para w = 0 hasta W-1 hacer
	Para h = 0 hasta H-1 hacer
		Para i = 0 hasta I-2 hacer
			Para j = 0 hasta K/2 hacer
				new_video[w][h].push_back(video[w][h][i])
			Fin para
			Para j = (K/2)+1 hasta K hacer
				new_video[w][h].push_back(video[w][h][i+1])
			Fin para
		Fin para
		new_video[w][h].push_back(video[w][h][I-1])
	Fin para
Fin para
Devolver new_video
\end{lstlisting}

\subsection{Interpolación lineal}

En este caso, usamos el polinomio interpolador de Lagrange entre cada par de puntos/pixeles consecutivos para aproximar los valores intermedios que irían en el video de camara lenta. Esto genera una función lineal para los pixeles consecutivos en la misma posición.

Por ejemplo, sean dos pixeles con valores 1 y 4:

\begin{bytefield}{8}
\begin{centering}
\bitbox{4}{1} & \bitbox{4}{4}
\end{centering}
\end{bytefield}

Si queremos un video en camara lenta con 5 cuadros intermedios por cada 2 del original, estos se replicarán de la siguiente forma:

\begin{bytefield}{28}
\begin{centering}
\bitbox{4}{1} & \bitbox{4}{1.5} & \bitbox{4}{2} & \bitbox{4}{2.5} & \bitbox{4}{3} & \bitbox{4}{3.5} & \bitbox{4}{4}
\end{centering}
\end{bytefield}

El procedimiento es el siguiente:

\begin{lstlisting}
Sean W,H,I el ancho, alto y la cantidad de frames del video original
Sea video[W][H][I] el triple vector de numeros enteros que representa el video original
Sea K la cantidad de frames que queremos agregar entre cuadro y cuadro
Crear un triple vector de enteros new_video[W][H][I+(I-1)*K]
Para w = 0 hasta W-1 hacer
	Para h = 0 hasta H-1 hacer
		Para i = 0 hasta I-2 hacer
			coef_cero = video[w][h][i]
			coef_uno = (video[w][h][i+1] - video[w][h][i]) / (K+1);
			Para k = 0 hasta K hacer
				pixel = coef_cero + coef_uno*k;
				Si (pixel < 0) pixel = 0
				Si (pixel > 255) pixel = 255
				new_video.push_back(pixel)
		Fin para
		new_video.push_back(video[w][h][I-1])
	Fin para
Fin para
Devolver new_video
\end{lstlisting}



