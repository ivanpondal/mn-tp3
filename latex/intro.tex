 El objetivo principal de este Trabajo Práctico es estudiar, implementar y analizar
 algoritmos de Interpolación para generar Videos con \textit{slow motion}.

Comenzaremos haciendo una breve introducción a los distintos métodos de Interpolación,
para luego explicar cual es el modelo que subyace en cada uno:
\begin{itemize}
    \item Interpolación por Vecinos.
    \item Interpolación Fragmentaria Lineal.
    \item Interpolación por Splines.
    \item Interpolación por Splines (bloques de tamaño fijo).
\end{itemize}

Una vez finalizada la parte del Modelo, pasaremos a describir la Implementación de los
diferentes métodos presentados, realizadas en \texttt{C++}.

Ya llegando al final, pasaremos a presentar la Experimentación realizada, a la vez
que iremos analizando y discutiendo los resultados obtenidos.

Los experimentos realizados pueden dividirse en dos categorías. La primera, relacionada
con el costo temporal y la correctitud de los diversos algoritmos utilizados:
\begin{itemize}
    \item Funcionamiento de los métodos implementados al tratar de interpolar diferentes
        familias de funciones.
    \item Comparación de tiempos de ejecución entre los distintos métodos.
    \item Determinación del tamaño de bloque del método Interpolación por Splines.
\end{itemize}

La segunda, relacionada con el aspecto cualitativo de los métodos:
\begin{itemize}
    \item Comparación del \textit{Error Cuadrático Medio} y \textit{Peak to Signal Noise Rate} entre
        los distintos métodos.
    \item Análisis del fenómeno de Artifacts.
\end{itemize}

Para finalizar, cerraremos el presente informe con una conclusión, en la cual
discutiremos acerca de los métodos vistos, así como de la experimentación realizada.
También, contaremos las dificultades encontradas al realizar el Trabajo Práctico,
las posibles continuaciones que se podrían realizar, y si los objetivos planteados
fueron alcanzados.
