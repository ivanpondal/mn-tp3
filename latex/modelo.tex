\subsection{Video}

Definiremos un modelo para los videos con el cual sea fácil de trabajar a la hora de realizar
el \textit{slow motion}.
Dado un video, definiremos:
\begin{itemize}
    \item $w$ el ancho en píxeles de cada frame.
    \item $h$ el ancho en píxeles de cada frame.
    \item $f_i$ el $i$-ésimo frame, con $0< i < k$, donde k es la cantidad de frames totales.
    \item $p(x,y,f_i)$, con $0 < x < w$, $0< y < h$, el píxel en la posición $(x,y)$ del frame $f_i$.
\end{itemize}

Luego, si tomamos $p(x,y,f_i)$ y $p(x,y,f_{i+1})$ querremos agregar una cierta cantidad
de píxeles entre ambos, de forma que haya una transición del primero al segundo y
se produzca el \textit{slow motion}.

Para elegir que valores agregar entre los píxeles, utilizaremos diferentes métodos
de interpolación.
\begin{itemize}
    \item \textit{Vecinos}: Consiste en rellenar los nuevos frames replicando los
        valores de los píxeles del frame original que se encuentra más cerca.
    \item \textit{Interpolación Lineal}: Consiste en rellenar los píxeles utilizando
        interpolaciones lineales entre píxeles de frames originales consecutivos.
    \item \textit{Interpolación por Splines}: Consiste en rellenar los píxeles utilizando
        Splines entre píxeles de frames originales consecutivos. En este método,
        utilizaremos la información provista por todos los frames del video,
        y generaremos $k-1$ funciones, cada una de a lo sumo grado cúbico.
    \item \textit{Interpolación por Splines con tamaño de bloque variante}:
        Simliar al anterior, pero con la posibilidad de variar la cantidad de frames
        tomados en cuenta al generar las funciones.
\end{itemize}

En las siguientes secciones explicaremos con mayor detalle cada uno de los métodos.

\subsection{Vecinos}\label{Vecinos}
En este método, eligiremos para cada nuevo píxel el valor del frame original que se encuentre
más cercano.
Si definimos $c$ como la cantidad de frames a agregar entre cada par original, y
$g_0, \dots, g_{c-1}$ los nuevos frames. Tenemos que:
\begin{equation*}
    \begin{aligned}
        p(x,y,f_{i}) &= p(x,y,f_{i})\\
        p(x,y,g_{0}) &= p(x,y,f_{i})\\
        \vdots\\
        p(x,y,g_{c/2-1}) &= p(x,y,f_{i})\\
        p(x,y,g_{c/2}) &= p(x,y,f_{i+1})\\
        \vdots\\
        p(x,y,g_{c-1}) &= p(x,y,f_{i+1})\\
        p(x,y,f_{i+1}) &= p(x,y,f_{i+1})
    \end{aligned}
\end{equation*}

\subsection{Interpolación Lineal}\label{Lineal}
En este método, buscaremos interpolar los píxeles de frames contiguos con una función
lineal. Para ello, construiremos un Polinomio Interpolante de grado 1 utilizando
\textit{diferencias divididas}, ya que ofrece una construcción más sencilla que al seguir
el método de Lagrange.

Luego, si llamamos $f$ a la función (desconocida excepto en los puntos $x_j$), definimos:
\begin{itemize}
    \item Diferencia dividida de orden cero en $x_j$:
        \begin{equation*}
            f[x_j] = f(x_j)
        \end{equation*}
    \item Diferencia dividida de orden uno en $x_j$, $x_{j+1}$:
        \begin{equation*}
            f[x_j,x_{j+1}]= \frac{f[x_{j+1}] - f[x_j]}{x_{j+1} - x_j} = \frac{f(x_{j+1}) - f(x_j)}{x_{j+1} - x_j}
        \end{equation*}
    \item Polinomio Interpolante de grado 1 para $x_j$, $x_{j+1}$:
        \begin{equation*}
            P_1(x) = f[x_j] + f[x_j,x_{j+1}](x-x_j) = f(x_j) + \frac{f(x_{j+1}) - f(x_j)}{x_{j+1} - x_j}*(x-x_j)
        \end{equation*}
\end{itemize}

\subsection{Interpolación por Splines}\label{Splines}

\subsection{Interpolación por Splines con tamaño de bloque variante}\label{MultiSplines}
